\documentclass[a4paper,12pt]{report}

\input{header.tex}


\title{Отчёт по курсовому проекту }
\author{(Спасенов Иван Владимирович, Доктор Артем Алексеевич)}

\begin{document}

\maketitle

\tableofcontents

\addcontentsline{toc}{chapter}{Введение}
\chapter*{Введение}

\section*{Серверная часть SMTP агента}

\subsection*{Задание. Вариант 10}

Используется вызов pselect и единственный рабочий поток. Журналирование в отдельном процессе. Нужно проверять обратную зону днс.

\subsection*{Цель и задачи}

Цель:
    Разработать \textbf{SMTP-сервер} с использованием одного потока и метода pselect().

Задачи:
\begin{itemize}
    \item проанализировать \textbf{SMTP}-протокол и разработать конечный автомат обработки SMTP-сообщений;
    \item реализовать программу для получения и сохранения писем по протоколу \textbf{SMTP} на языке программирования \textbf{C};
    \item оформить расчетно-пояснительную записку.
\end{itemize}

\section*{Клиентская часть SMTP агента}

\subsection*{Задание. Вариант }


\chapter{Аналитический раздел}

\section{Предметная область}

\subsection{ER-диаграмма предметной области}

В результате проведенного исследования были выявлены следующие сущности предметной области:

\begin{enumerate}
	\item Клиент.
	\item Сервер.
	\item Логгер.
	\item Письмо.
    \item Отправитель.
    \item Получатель.
    \item Данные письма.
\end{enumerate}

Зависимость между сущностями предметной области может быть описана ER-диаграммой (~\ref{fig:er_diagram} ).
\begin{figure}
    \centering
    \includegraphics[width=\textwidth]{../images/er.png}
    \caption{ER-диаграмма предметной области}
    \label{fig:er_diagram}
\end{figure}

\newpage

\section{Достоинства и недостатки реализуемой архитектуры}

\subsection{Серверная часть SMTP агента}

Согласно условию задачи, в работе сервера предлагается использовать один поток выполнения и один отдельный поток
журналирования.

Достоинства варианта реализации:
\begin{itemize}
    \item простота реализации, отсутствует необходимость реализации разделяемой памяти и взаимодействия между процессами или потоками;
    \item отсутствие времени на переключение контекстов;
    \item благодаря неблокирующему вводу/выводу, сервер может обслуживать множество клиентов с достаточно высокой производительностью, при условии, что обработка занимает мало времени;
    \item логирование в отдельном процессе позволяет не блокироваться на операциях ввода/вывода при записи в файл или в терминал;
\end{itemize}

Недостатки данной архитектуры:
\begin{itemize}
    \item низкая производительность при длительной обработке клиентских команд;
    \item низкая отказоустойчивость (использование одного потока является менее надежным при возникновении фатальных ошибок в приложении, чем при наличии нескольих взаимозаменяемых потоков,);
    \item сложность масштабирования и использования всех аппаратных ресурсов системы.
\end{itemize}

Недостатки программной реализации с одним потоком выполнения и мультиплексированием можно уменьшить с  помощью
создания нескольких (пула) потоков с неблокирующим вводом/выводом и распределения нагрузки между ними.

\subsection{Клиентская часть SMTP агента}

\chapter{Конструкторский раздел}

\section{Конечный автомат состояний сервера}

На рис. ~\ref{fig:serverfsm} представлен сгенерированный с использованием \textit{fsm2dot} скрипта из \textit{autogen}
файла конфигурации конечного автомата \textit{serverfsm.def} и \textit{dot}.

\begin{figure}
    \centering
    \includegraphics[width=\textwidth]{../images/serverfsm.png}
    \caption{Построенный граф конечного автомата SMTP сервера}
    \label{fig:serverfsm}
\end{figure}

\newpage


\section{Синтаксис команд протокола}
Ниже приведен формат команд сообщений протокола в виде регулярных выражений:

Регулярные выражения SMTP команд:
\begin{description}
    \item[NOOP]
    \texttt{[Nn][Oo][Oo][Pp]\textbackslash{}\textbackslash{}r\textbackslash{}\textbackslash{}n}
    \item[HELO]
    \texttt{[Hh][Ee][Ll][Oo]\textbackslash{}\textbackslash{}s*(?<domain>.+)\textbackslash{}\textbackslash{}r\textbackslash{}\textbackslash{}n}
    \item[EHLO]
    \texttt{[Ee][Hh][Ll][Oo]\textbackslash{}\textbackslash{}s*(?<domain>.+)\textbackslash{}\textbackslash{}r\textbackslash{}\textbackslash{}n}
    \item[MAIL]
    \texttt{[Mm][Aa][Ii][Ll] [Ff][Rr][Oo][Mm]:\textbackslash{}\textbackslash{}s*<(?<address>.+@.+)?>\textbackslash{}\textbackslash{}r\textbackslash{}\textbackslash{}n}
    \item[RCPT]
    \texttt{[Rr][Cc][Pp][Tt] [Tt][Oo]:\textbackslash{}\textbackslash{}s*<(?<address>.+@.+)>\textbackslash{}\textbackslash{}r\textbackslash{}\textbackslash{}n}
    \item[VRFY]
    \texttt{[Vv][Rr][Ff][Yy]\textbackslash{}\textbackslash{}s*(?<domain>.+)\textbackslash{}\textbackslash{}r\textbackslash{}\textbackslash{}n}
    \item[DATA]
    \texttt{[Dd][Aa][Tt][Aa]\textbackslash{}\textbackslash{}r\textbackslash{}\textbackslash{}n}
    \item[RSET]
    \texttt{[Rr][Ss][Ee][Tt]\textbackslash{}\textbackslash{}r\textbackslash{}\textbackslash{}n}
    \item[QUIT]
    \texttt{[Qq][Uu][Ii][Tt]\textbackslash{}\textbackslash{}r\textbackslash{}\textbackslash{}n}
    \item[Окончание данных письма]
    \texttt{\^{}\textbackslash{}\textbackslash{}.\textbackslash{}\textbackslash{}r\textbackslash{}\textbackslash{}n}
\end{description}

\section{Синтаксис команд протокола}
На рис.~\ref{fig:uml_server_ph} и на рис.~\ref{fig:uml_server_log} представлены физическая и логическая
диаграммы представления данных в системе соответственно.

\begin{figure}
    \centering
    \includegraphics[width=\textwidth]{../images/uml_server_ph.png}
    \caption{Физическая диаграмма представления данных в серверной части системы}
    \label{fig:uml_server_ph}
\end{figure}

\begin{figure}
    \centering
    \includegraphics[width=\textwidth]{../images/uml_server_log.png}
    \caption{Логическая диаграмма представления данных в серверной части системы}
    \label{fig:uml_server_log}
\end{figure}


Конфигурация \textit{serveropts.def}, используемая для автоматической генерации исходного кода обработки
входных флагов приложенияс помощью \textit{autogen}.:

% \lstset{language=C}
% \lstinputlisting{../../source/server/autogen/serveropts.def}

\chapter{Технологический раздел}

\section{Сборка программы}

Сборка SMTP агента состоит из трех \textit{Makefile} системы сборки \textit{make}:

\begin{itemize}
    \item сборка клиента,
    \item сборка сервера,
    \item сборка отчета.
\end{itemize}

\subsection{Сборка серверной части SMTP агента}

Сборка SMTP сервера состоит из следующих целей:

\begin{enumerate}
    \item сборка сервера;
    \item сборка тестового клиента;
    \item генерация исходных кодов конечного автомата и опций с помощью \textit{autogen};
    \item сборка сервера и тестового клиента;
    \item запуск системного тестирования.
\end{enumerate}

Сборка программы осуществляется с помощью следующей команды:
\begin{verbatim}
    make autogen_all && make all
\end{verbatim}


\addcontentsline{toc}{chapter}{Выводы}

\chapter*{Выводы}

\section{Серверная часть SMTP агента}

В результате выполнения курсового проекта была достугнута поставленная цель, а именно разработан
\textbf{SMTP-сервер} с использованием одного потока и метода pselect(), осуществляющее прием и сохранение писем
для дальнейшей поставки их пользователям.

Во время выполнения работы были выполнены следующие задачи:
\begin{itemize}
    \item проанализировано архитектурное решение, данное по условиям задачи, определены его преимущества и недостатки;
    \item разработан и реализован подход для обработки входящих соединений на основе метода pselect();
    \item разработан и реализовано хранение входящих писем в каталоге maildir;
    \item проанализирован протокол \textbf{SMTP} и реализован конечный автомат обработки входящих SMTP-сообщений;
\end{itemize}

А также получены и закреплены следующие навыки:
\begin{itemize}
    \item проектирование и реализация сетевого протокола SMTP;
    \item реализация серверного приложения с несколькими процессами на языке программирования Си;
    \item создание сценариев сборки программного обеспечения;
    \item использование latex и сценариев сборки для автогенерации расчетно-пояснительной записки.
\end{itemize}

В ходе работы не были реализованы следующие пункты, планируемые к разработке в дальнейшем:
\begin{itemize}
    \item проверка обратной зоны DNS;
    \item генерация документации и графов функци с помощью утилит;
    \item использование внешних конфигурационных файлов;
    \item модульное тестирование;
\end{itemize}


\end{document}
